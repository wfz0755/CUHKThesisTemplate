%
% chapter background study

\chapter{Background Study}

% classification - static vs dynamic
%
Casavant and Kuhl \cite{casavant88} classified load distribution
algorithms into static algorithms and dynamic algorithms.
In {\it static algorithms\/}, task
assignment decisions are made {\it a priori\/} during compilation time and
are remain unchanged during run-time. In contrast, {\it dynamic algorithms\/}
attempt to use current system workload information for run-time assignment
of tasks to the appropriate nodes.


\section{Suboptimal}
% suboptimal heuristic-based
%
It is now commonly agreed that despite the higher run-time complexity,
dynamic algorithms can potentially provide better performance than
static algorithms.
An important category of dynamic algorithms under
Casavant and Kuhl's taxonomy
is the {\it dynamic suboptimal heuristic-based algorithms\/}.
The majority of existing dynamic LD algorithms fall
into this category
\cite{eager86a,feng94,lu95b,lionel85,shiv90,shiv92,shiv95,wang85}.
Due to the intrinsic unavailability of accurate
and timely global state information in a distributed system,
targeting for suboptimal performance using heuristic-based
algorithms is quite reasonable.


\subsection{Components of Algorithms}
% components of dynamic algorithms
%
The major components of a dynamic LD algorithm are:
(1) the {\it location policy\/}, referring to the strategy used to
search for a task transfer partner;
(2) the {\it information policy}, referring to the way load
information is disseminated among processing nodes; and
(3) the {\it transfer policy\/}, referring to the strategy used to
determine whether task transfer activities should be initiated by a node,
either as a task sender or a task receiver.

\section{Location Policy}

% location policy - polling
%
Most existing location policies are {\it polling-based\/}.
A polling session can be started by a heavily loaded node in
an attempt to locate a lightly loaded task receiver, or {\it vice versa\/}.
The polling session in the former case is said to be
{\it sender-initiated\/} and the
latter to be {\it receiver-initiated\/} \cite{eager86b,wang85}.
These two basic approaches can be combined to form a
{\it symmetri\-cally-initiated\/} algorithm,
where polling sessions can be started by both senders and receivers.
%
A study on dynamic LD algorithms by Eager {\it et al\/} \cite{eager86b}
showed that neither pure sender-initiated
nor pure receiver-initiated location policy performs
consistently over the whole range of system workload.
%
In~\cite{shiv90}, Shivaratri and Krueger
proposed an adaptive symmetri\-cally-initiated location policy
so that sender-initiated pollings only occur
at low system workload, whereas receiver-initiated pollings are
conducted whenever appropriate.
This location policy shows performance advantage over a wide range
of system workload.
An important property of this algorithm is that load information
gathered during polling activities are retained so as to provide
a heuristic guide for subsequent polling activities.


% load index - number of tasks in homogeneous systems
%              our virtual load and its predictive behavior
%
A {\it workload index\/} measures how busy a node is.
The most commonly used workload index is
the number of application tasks residing in a node \cite{hatch90,kremien92}.
Most existing transfer policies are {\it threshold-based\/},
which means that a node qualifies as a task sender if its workload index
exceeds certain threshold value
\cite{eager86a,lu94,mirchandaney90,shiv90,stankovic84}.
In contrast, a node qualifies as a task receiver if
its workload index is below the threshold.
When the workload index equals to the threshold,
the node is regarded as ``normally'' loaded and no task transfer activity
is needed.


\chapterend
