%
% chapter What is an Operating System?

\chapter{What is an Operating System?}

An {\it operating system\/} is a program that acts as an 
\textit{intermediary\/} between a user of a computer and the computer hardware. 
The purpose of an operating system is to provide the environment in which
a user can execute programs. The primary goal of an operating system
is thus to make the computer system \underline{\textbf{\textit{convenient}}}
to use. A secondary goal is to use the computer hardware in
an \underline{\textbf{\textit{efficient}}} manner.

%\vfill
%
%{
%\noindent
%\underline{\bf Textbook}  \\[3mm]
%\noindent {\it ``Operating System Concepts'' } \\[2mm]
%\noindent Silberschatz Galvin \\[2mm]
%\noindent Fifth Edition, Addison Wesley
%}


%=======================================================
\newpage

\begin{figure}
\centerline{\psfig{figure=./whatis/level.eps,width=13cm}}
\caption{Abstract view of the components of a computer system.}
\label{fig:level}
\end{figure}

\vskip 1cm
\begin{itemize}
\item   A computer system can be divided roughly into four components:
        the hardware, the operating system, the application programs,
        and the users.

        \begin{itemize}
        \item   The hardware --- CPU, memory, I/O devices, etc.
                --- provides the basic computing resources.
        \item   The applications --- compilers, database systems, games,
                business programs --- defines the ways in which
                the hardware resources are used to solve the computing
                problems for the users.
        \end{itemize}

\item	Examples
	\begin{itemize}
	\item	Mainframe computers --- IBM OS/360, ...
	\item	Minicomputers --- Unix (HP-UX, Ultrix, AIX),
		OS/400, HP MPV, VAX OS, AOS, ... 
	\item	Personal computers --- CP/M, MS-DOS, MS-Windows, Mac OS,
			Linux, BeOS, ...
	\item	Palm tops --- Windows-CE, Palm OS, ...
	\end{itemize}
				

\end{itemize}



%=======================================================
\newpage

\note{Roles of an Operating System}{
The fundamental goal of computer systems is to execute user programs
and to make solving user problems easier.
Since bare hardware alone is not particularly easy to use,
application programs are developed. These various programs require
{\it certain common operations\/}, such as those controlling the I/O devices.
The common functions of controlling and allocating resources
are then brought together into one piece of software:
the operating system \cite{hsieh96} \cite{hsieh97}.

\begin{itemize}
\item   An operating system is similar to a {\it government\/}
        --- The operating system controls and coordinates the use of the
        hardware among the various application programs for the
        various users.

\item   An operating system can be viewed as a {\it resource allocator\/}
        --- The operating system acts as the manager of the resources
        and allocates them to specific programs and users as necessary.

\item   An operating system is a {\it control program\/}
        --- The operating system controls the execution of user programs
        to prevent errors and improper use of the computer.

\item   In general,
        there is no adequate definition of an operating system.
\end{itemize}
}


