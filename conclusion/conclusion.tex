%
% chapter conclusion

\chapter{Conclusion}


A load distribution algorithm based on 
anti-tasks and load state vectors is proposed.
Anti-tasks are composite agents which travel around the distributed
system to facilitate the pairing up of task senders and receivers 
as well as the collection and dissemination of load information.
Time-stamped load information is stored 
in load state vectors which when used with anti-tasks,
encourages mutual sharing of load information among processing nodes.
The most important property of the algorithm is that
anti-tasks are spontaneously directed towards processing nodes
with high transient workload, thus allowing their surplus workload
to be relocated quickly.

We found from simulation experiments that our algorithm provides 
significant reduction of mean task response time over a large range 
of system sizes.
The cost for achieving this performance gain in terms of
CPU overhead and channel bandwidth consumption is generally
comparable to the other algorithms we studied.
%
%To further our understanding on the performance characteristics
%of the anti-task algorithm,
%we plan to experiment the algorithm 
%on the testbed developed by Lu and Hsieh.

On the other hand, 
we are currently working on an improvement of the anti-task algorithm
by employing some {\it proxy nodes\/} in the distributed system 
to store load state information. 
Each proxy node serves as a mediator to collect and distribute
load state information for a subset of nodes.
By visiting a proxy node, an anti-task does not need to visit all
the nodes in the system in order to acquire load information of
the whole system.
By then, the CPU overhead and communication cost can be limited.


\chapterend
