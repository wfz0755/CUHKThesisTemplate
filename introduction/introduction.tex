%
% chapter introduction

\chapter{Introduction}


Most existing LD algorithms are {\it polling-based\/}.
It means that a node attempting to identify a task transfer partner
will send a query message to a selected target node to ask for its consent
\cite{casavant88,eager86a,eager86b,lu94,lu95b,shiv90,wang85}.
The two negotiating nodes involved may share load information of each other
by explicitly declaring
their load states in the polling and reply messages.
Alternatively, load information can be deduced from the
semantics of messages exchanged.
%
As pollings are one-to-one in nature,
a severe weakness of polling-based LD algorithms is that
load information exchanged during a polling session
is confined to the two negotiating nodes only.
Consequently, to spread out load information among the constituent nodes,
a lot of polling activities need to be conducted.

In most polling-based LD algorithms, there is a
{\it polling limit\/} to control the number of polling trials that a node
can make during each attempt to locate a transfer partner.
Polling limits are usually set to $f \cdot N$, where $N$ is the number
of nodes in the DCS and $f$ is a fractional constant
\cite{eager86a,lu95b}.
%
Another weakness of polling-based LD algorithms relates to
the use of polling limits.
When the DCS grows large (in terms of $N$),
more and more polling trials are required before a node can successfully
pair up a suitable task transfer partner.
%
Accompanied with the larger number of pollings is
a higher amount of network bandwidth consumption and CPU overhead.
If these algorithm execution costs are not controlled carefully,
the performance penalty imposed may break even the
performance gain obtained via load distribution.
In the worse situation, the performance of the system
running the LD algorithm may be worse than the performance of 
a system without load distribution.
We say that polling-based algorithms are {\it not scalable\/}.


\chapterend
